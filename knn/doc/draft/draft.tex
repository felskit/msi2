\documentclass[12pt,a4paper]{article}
\usepackage{fullpage}
\usepackage[T1]{fontenc} 
\usepackage[utf8]{inputenc}
\usepackage{amsmath}
\usepackage{amssymb}
\usepackage[hidelinks]{hyperref}
\usepackage[polish]{babel}

\begin{document}

\title{Metody sztucznej inteligencji 2 \\
\Large{
    Projekt 1. --- Algorytm $k$ najbliższych sąsiadów \\
    Konspekt
}}
\author{Bartłomiej Dach, Tymon Felski}
\maketitle

Niniejszy dokument zawiera informacje wstępne dotyczące pierwszego projektu, którego celem jest zaimplementowanie algorytmu $k$ najbliższych sąsiadów ($k$-NN, ang. \emph{$k$ nearest neighbors}) i~analiza jego działania dla~dostarczonych danych treningowych.

\section{Skład grupy}

Projekt realizowany będzie w~dwuosobowej grupie, w~składzie:

\begin{enumerate}
    \item Bartłomiej Dach,
    \item Tymon Felski.
\end{enumerate}

\section{Opis algorytmu}

\section{Wybrane technologie}

\begin{thebibliography}{9}

    \bibitem{hastie2009}
        T. Hastie,
        R. Tibshirani,
        J. Friedman,
        \emph{The Elements of Statistical Learning: Data Mining, Inference, and Prediction}.
        Nowy Jork: Springer-Verlag,
        2009.
        [Online] \\
        Dostępne: \url{https://web.stanford.edu/~hastie/ElemStatLearn/}.
        [Dostęp 26~lutego 2018]

    \bibitem{vapnik1998}
        V.N. Vapnik,
        \emph{Statistical learning theory}.
        Nowy Jork: John Wiley and Sons,
        1998.

\end{thebibliography}

\end{document}